In order to determine the necessary amount of data for accurate estimation of the functional connectivity matrix, we calculate Fisher information for $P(W|H)$. Assuming for simplicity perfect knowledge of spike trains (i.e. such not corrupted by inference errors from calcium imaging) and single time-bin coupling $m=1$, i.e. $w_{ij}(t)\neq 0$ only for time-delay $t=1$, we write the Fisher information matrix as

\begin{equation}
\begin{array}{rl}
C^{-1}=\frac{\partial (-\ln P)}{\partial w_{ij}\partial w_{i'j'}}
=-&\delta_{ii'}\sum\limits_t\left[
n_i(t)n_{j}(t-1)n_{j'}(t-1)\left(-\frac{f'(J_i(t))^2}{f(J_i(t))^2} +
\frac{f''(J_i(t))}{f(J_i(t))}\right) \right. \\
&\left.-\Delta (1-n_i(t))n_{j}(t-1)n_{j'}(t-1)f''(J_i(t))\right].
\end{array}
\end{equation}

For exponential transfer function $f(J)$ and assuming weak coupling between spikes this may be rewritten as

\begin{equation}\label{eqn:fisher}
\begin{array}{rl}
C^{-1}
&=\delta_{ii'} (T\Delta) P(n_i(t)=0, n_j(t-1)=1, n_{j'}(t-1)=1)E[f(J_i(t))|n_i(t)=0, n_j(t-1)=1, n_{j'}(t-1)=1] \\
&\sim (T\Delta)\left[(r \tau_w)\delta_{ii'}\delta_{jj'}+O((r \tau_w)^2)\right]r.
\end{array}
\end{equation}

Here $(T\Delta)$ is the total observation time, $ \tau_w$ is ``the coincidence time'' - the typical EPSP/IPSP time-scale over which the spike of one neuron affects the spike probability of the other neuron, and $r \approx E[f(J_i(t))|n_i(t)=0, n_j(t-1)=1, n_{j'}(t-1)=1]$ is the typical firing rate.  For successful determination of the functional connectivity weights $W$, the variance $C$ should be smaller than the typical scale $\langle W^2\rangle$, i.e.

\begin{equation}
(T\Delta) \sim (W^2 r^2  \tau_w)^{-1}.
\end{equation}

For typical values of $W^2\approx 0.1$, $r\approx 5$ Hz and $ \tau_w \approx 10$ ms, 
with this order of magnitude estimate we obtain $T$ of the order of hundred seconds.
This theoretical estimate of the necessary amount of fluorescent data is in good agreement with our simulations below.

Note also that necessary recording time does not depend on the number of neurons in the imaged network $N$. This, at first, unexpected result is the direct consequence of the special form of $C^{-1}$ in Eq.\eqref{eqn:fisher}. In particular, when $r \tau_w <<1$, this matrix is dominated by the diagonal term $(T\Delta)(r^2  \tau_w)$, and so the Fisher information matrix is predominantly diagonal with the scale $(r^2 \tau_w T\Delta)^{-1}$, independent of the number of neurons $N$. This theoretical result is also directly confirmed in our simulations below.
