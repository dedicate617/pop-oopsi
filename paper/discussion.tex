Functional connectivity may fail to faithfully represent anatomical circuit structure if false correlations are present between different neurons, induced e.g. by common inputs, or if the dynamics of neural population is entirely concentrated on a low-dimensional subspace of the full configurational space ${\bf n}$. Note that these two statements are, in a sense, stating the same condition: if activity of different neurons is tightly correlated, their dynamics is concentrated on a low-dimensional plane and vice-versa - concentration of dynamics onto a low-dimensional plane will be perceived as correlation in activity of different neurons. In turn, low dimensionality of the neural dynamics may be caused by different factors, including common input, small subset of command neurons driving the circuit, or even emergent property of a network. Low dimensionality of neural dynamics results in that the inference problem becomes underdetermined, i.e. there may exist directions in ${\bf w}_i$ along which connectivity is not constrained by neural activity data (i.e. directions orthogonal to the subspace of all observed neural activity configurations), or is poorly constrained. This, naturally, leads to ${\bf w}_i$ being poorly defined along these directions. The necessary condition for good correspondence between functional connectivity weights ${\bf w}_i$ and anatomical connectivity, therefore, is {\em full-dimensionality} of the observed set of neural configurations. In case of spontaneously firing system of neurons this condition is satisfied by many neuron-firings occurring independently, thus, allowing to fully sample all possible directions in ${\bf w}_i$.  Still, spontaneously active preparation by itself may fail to display sufficient degree of independence between firing of neurons due to low-dimensionality of observed activity space, e.g. because of emergent properties of the circuit. In that case necessary variety of independent neural activity patterns may be enforced by randomly activating subsets of neurons via ChR2 or glutamate uncaging.

We also note that the correlations induced by secondary and so on synaptic transmissions (such as when neuron $A$ results in firing of neuron $B$, which in turn results in firing by neuron $C$), are all properly resolved in GLM-fitting process via the so called explaining-away process. In other words, because we do not just identify correlations between neural firings with the functional connectivity weights $w_{ij}$, but instead statistically fit a model of neural interactions, if found weights between neurons $A$ and $B$, and $B$ and $C$ are sufficient to explain the correlation between $A$ and $C$, the weight connecting $A$ and $C$ will not appear in the model - the correlation between $A$ and $C$ was ``explained away'' by correlations between $A$ and $B$, and $B$ and $C$. By this, the multi-synaptic firing patterns do not confuse our estimation process.

ADD SOME RAVINGS ABOUT PROPER/IMPROPER FUNCTIONAL CONNECTIVITY.
