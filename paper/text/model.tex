\subsection{Single neuron model}

We assume that we have a discrete time model, with time step size $\Delta$ and $T$ total time steps. In each time step, the neuron can emit any non-negative number of spikes, i.e., $n_t \in \{0,1,2,\cdots\}=\mathbb{N}_0$.  The spiking is governed by a Poisson process with rate $\lambda_t \Delta$, where $\lambda_t$ is given by: 

\begin{align} \label{eq:1neur}
\lambda_t &= \exp\{b + \ve{k}' \ve{s}_t + \omega h_t\} \\
h_t - h_{t-1}&= - \frac{\Delta}{\tau_h} h_t + n_{t-1} + \sigma_h \sqrt{\Delta} \varepsilon
\end{align} 

%\noindent where $b$ sets the baseline firing rate, $\ve{k}$ is a linear filter operating on the external stimulus, $\ve{s}_t$, $\omega$ is the weight on the spike history term, $h_t$. The spike history term jumps instantaneously after a spike, and then decays with time constant $\tau_h$ back to zero. Spiking induces a change in the intracellular calcium concentration, $\Ca$, which is measured using fluorescence, $F_t$: 

calcium model:

\begin{align}
\Ca_t - \Ca_{t-1} &= -\frac{\Delta}{\tau_c}(\Ca_t + \Ca_b) + A n_t + \sigma_c \sqrt{\Delta} \varepsilon, 
\end{align}

observation model:
\begin{align} \label{eq:1obs}
F_t = \alpha S(\Ca_t) + \beta + (S(\Ca_t)+\sigma_F) \varepsilon_t
\end{align} 

\noindent where $S(x)=x^n/(x^n+k_d)$ is the standard Hill equation,  the $\varepsilon$'s indicate standard normal random variables.


\subsection{Multiple neuron model} \label{sec:popmod}

We assume the firing rate for neuron $i$, $\lambda_{i,t} \Delta$, is given by:

\begin{align} \label{eq:poisson}
\lambda_{i,t} &= f(b_i + \ve{k}_i'\ve{s}_t + \sum_{j=1}^N \omega_{ij} h_{j,t}) \\
h_{i,t}-h_{i,t-1} &= -\frac{\Delta}{\tau_{h_i}} h_{i,t} + n_{i,t-1}  + \sigma_{h_i} \sqrt{\Delta} \varepsilon,
\end{align} 

\noindent where each $h_{i}$ corresponds to the spike history term associated with neuron $i$, and has an associated time-constant, $\tau_i$.  The input each neuron recieves, therefore, is the sum of spike history terms from all neurons (including itself).  Thus, $\ve{\omega}$ = $\{\omega_{ij}\}$ = $\{\omega_{11}$, $\omega_{12}$, \ldots, $\omega_{1N}$, $\ldots$, $\omega_{NN}\}$ comprises the weight matrix for these neurons.

calcium model:

\begin{align}
\Ca_{i,t} - \Ca_{i,t-1}&= -\frac{\Delta}{\tau_{c_i}} (\Ca_{i,t} + \Ca_{b_i}) + A_i n_{i,t} + \sigma_{c_i} \sqrt{\Delta} \varepsilon, 
\end{align}

observation model:
\begin{align} \label{eq:2obs}
F_{i,t} &= \alpha_i S(\Ca_{i,t}) + \beta_i + (S(\Ca_{i,t}) + \sigma_{F_i}) \varepsilon,
\end{align} 

Note that we assume that $S(\cdot)$ is the same for each neuron (i.e., the nonlinear function is the same for each neuron, because the parameters are a function of the indicator, not the neuron).  Further note that we assume (for now) that the noise on the calcium and observation for each neuron is independent (an assumption that we will probably want to relax later).
