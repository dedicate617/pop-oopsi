Our goal is to estimate the model parameters, such as $W$ and $M$, given calcium imaging observations ${\bf F}$. A natural approach for such estimation is to obtain MAP estimate of these parameters with

\begin{equation}\label{eqn:MAP}
(M, W)^*=\text{argmax } P(M, W|{\bf F}).
\end{equation}

While we may be unable to calculate the probability $P(M, W|{\bf F})$, our hierarchical model Eqs. \eqref{eqn:glm:definition} and \eqref{eqn:ca:definition} allows us to calculate $P({\bf F}, {\bf X}, M, W)\sim P({\bf F}, {\bf X}|M, W)$. A natural approach to solving problem (\ref{eqn:MAP}) then is to use the Expectation Maximization algorithm (EM).  EM allows to obtain MAP estimates of HMM parameters by going through a series of iterations $(M, W)^{(l)}$, each time finding a better approximation to the MAP solution by maximizing the conditional expectation values

\begin{equation}
(M, W)^{(l+1)}=\text{argmax }E_{P({\bf X}|M^{(l)}, W^{(l)}, {\bf F})}\left[\log P({\bf X}, {\bf F}, M, W)\right].
\end{equation}

The expectation to be computed in our case is
\begin{equation}\label{eqn:loglik:definition-expl}
\sum_i^N E[\log P_{[Ca]}(F_i|{\bf X}, M_i, W)] + \sum_i^N E[\log P_{GLM}(n_i|{\bf n}_{\i}, W)].
\end{equation}

Here ${\bf n}_{\i}=(n_1, ..., n_{i-1}, n_{i+1}, ...n_N)$, $P_{[Ca]}(F_i|{\bf X}, M_i, W)$ is the likelihood of observing signal $F_i$ \cite{Vogelstein2009}, and $P_{GLM}$ likelihood can be calculated from Eq. \eqref{eqn:glm:definition}), 

\begin{equation}\label{eqn:likelihoodGLM-expl}
\begin{array}{l}
E\left[\log P_{GLM}(n_i|{\bf n}_{\i}, W\right]=\sum\limits_t E \left[ n_i(t) J_i(t)\right] - E\left[(1-n_i(t)) \exp(J_i(t)) \Delta \right], \\
J_i(t)=b_i+\sum\limits_{j} \sum\limits_{t'<t} w_{ij}(t-t')n_{j}(t'), 
\end{array}
\end{equation}

Note that while the first term in Eq. \eqref{eqn:likelihoodGLM-expl} is simple to compute and only requires $E[n_i(t)]$ and $E[n_i(t) n_{j}(t')]$, the second term does not reduce to simple sufficient statistics. In order to evaluate this expectation value we need to obtain joint sample from the distribution of all spike trains $P({\bf n}|M, W, {\bf F})$.

Given Markovian nature of model Eq. \eqref{eqn:glm:definition}) and (\ref{eqn:ca:definition}), obtaining such sample is an instance of a well known problem of sampling from HMM. In particular, for a finite state-space HMM forward-backward procedure is known to provide samples in  $O(T)$ time \cite{RAB89}. For continuous state-space HMM different sampling strategies exist relying on discretizing the continuous state-space and approximating the integrals in forward-backward procedure using Monte Carlo \cite{DFG01, MINKAPHD, Fearnhead2003, koyama08, Andrieu2007, NBR03}.

In our case, the state $X_i(t)$ is a direct product of binary $n_i(t)$ and continuous $C_i(t)$ dimensions, and so a continuous state-space sampling algorithm is required.
Given large length of neural activity recordings data, $O(T)$ computational cost of HMM sampling algorithms is an important advantage.

One of the most popular methods for sampling from continuous-state HMM is sequential Monte Carlo (SMC), also known as particle filter \cite{DFG01}. In SMC, discretization of the state-space is constructed by drawing a sample from marginal distributions  $P({\bf X}(t)|\{{\bf F}(t'), t'=1\ldots t\})$, computed in the forward pass, and the forward-backward integrals are approximated using Monte Carlo on such samples. The main difficulty of SMC in our case is extremely high dimensionality of the state-space - for a population of $N\sim 50$ neurons, the dimensionality of the state space is $2N\sim 100$. Integrals in forward-backward procedure, therefore, cannot be accurately sampled using particle swarms of tractable size.

To solve this problem, we propose a hybrid SMC-Gibbs sampling strategy taking advantage of the specific structure of the model Eqs. \eqref{eqn:glm:definition} and (\ref{eqn:ca:definition}, namely, that it can be viewed as a set of $N$ coupled HMM models. Gibbs sampling is a procedure for obtaining samples from high-dimensional distribution $P({\bf X})$ by sampling from low-dimensional conditional distributions $P(X_{i}(t)|{\bf X}_{\i, t})$ one variable $X_{i}(t)$ at a time \cite{Gelfand1990}.  Gibbs sampling allows to reduce intractable high-dimensional sampling problems to sequences of tractable low-dimensional subproblems.  Here, we propose Gibbs sampling in blocks of one neuron sequence $X_{i}$ at a time: if we view the spike train history ${\bf X}=(X_1, ..., X_N)$ as a set of blocks of spike trains from individual neurons $X_i$, we perform Gibbs sampling by consequently sampling such entire blocks $X_i\sim P(X_{i}|{\bf X}_{\i}, M, W, {\bf F})$ one at a time.  Note that such block-sampling strategy is necessary here since different $t$ states in $X_i(t)$ for same neuron $i$ are correlated via Markov dynamics, thus leading to slow mixing of Gibbs chain if we Gibbs-sample from $X_i(t)$ for different $t$ and same $i$.  Although each sampling sub-problem in our case is still high-dimensional, it is tractable because it reduces to sampling from HMM with 2D state-space. Thus, joint sample from $2N$-dimensional HMM $P({\bf X}|M, W, {\bf F})$ may be obtained.

Maximization step of EM requires maximizing conditional expectation of $\log P({\bf X}, {\bf F}, M, W)$ given such sample. In our case, such maximization is a very large optimization problem involving $6N$ parameters $M$ and $mN^2+N$ parameters $W$. Fortunately, special structure of Eq. \eqref{eqn:loglik:definition-expl}) allows to simplify optimization problem dramatically by performing optimization with respect to different neurons $i$ independently.

EM algorithm therefore requires solving following problems: sampling individual neuron state-sequences $X_i\sim P(X_{i}|{\bf X}_{\i}, M, W, {\bf F})$ using HMM sampling technique, obtaining joint sample of neuron state-sequences ${\bf X}\sim P({\bf X}|M, W, {\bf F})$ using Gibbs technique, solving for next iteration of parameters $M$ and $W$ by maximizing conditional expectation value of the log-likelihood $P({\bf X}, {\bf F}, M, W)$.

\begin{algorithm}
\caption{Pseudocode for estimating functional connectivity from calcium imaging data using EM.}\label{eqn:pseudocode}
\begin{algorithmic}
\While{$|W^{(l)}-W^{(l-1)}|>\text{threshold}_W$}
  \ForAll{$i=1\ldots N$}
    \While{$|M^{(l)}_i-M^{(l-1)}_i|> \text{threshold}_M$}
      \State Sample $X_i \sim P(X_i|{\bf n}_{\i}, M_i, W, {\bf F}_i)$
      \State Maximize $M^{(l+1)}=\text{argmax} E\left[\log P_{[Ca]}(X_i, {\bf n}_{\i}, M_i, W, {\bf F}_i)|X_i\right]$
    \EndWhile
  \EndFor
  
  \For{$k=1\ldots N_G$}
    \ForAll{$i=1\ldots N$}
      \State Sample $n_i \sim P(n_i|{\bf n}_{\i}, M_i, W, {\bf F}_i)$
    \EndFor
  \EndFor 

  \State Maximize $W^{(l+1)}=\text{argmax} E\left[\log P_{GLM}({\bf n}|W)|{\bf n}\right]$  
\EndWhile
\end{algorithmic}
\end{algorithm}
