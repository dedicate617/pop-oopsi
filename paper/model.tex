In this work neural activity of a population of neurons is modeled with the standard generalized linear model (GLM), which is known to capture well the statistical properties of the firing of individual neurons \cite{PILL07, PAN03d, Wu07, Rigat06, OKA05}, 

\begin{equation}\label{eqn:glm:definition}
\begin{array}{l}
n_i(t)\sim \text{Bernoulli}(\lambda_i(t)\Delta), \\
\lambda_i(t)=f(J_i(t))=f(b_i+k_i\cdot S(t)+\sum\limits_{j=1}^{j=N} \sum\limits_{t'<t} w_{ij}(t-t')n_{j}(t')).
\end{array}
\end{equation}

($ 0 < \lambda_i(t)\Delta<1$)
Here, the spiking of neurons is described as a stochastic Bernoulli process driven by external stimulus $s(t)$ as well as the activity of other neurons in the population ${\bf n}=\{n_{j}(t)\}$, connected to each other via synapses $w_{ij}$.  $\lambda_i(t)$ is instantaneous firing rate of neuron $i$, $b_i$ and $k_i\cdot S(t)$ are the baseline and the external stimulus terms, respectively, and $\Delta$ is the time step.  In this paper we are concerned with studying spontaneous activity in a population of neurons, thus the external stimulus term $S(t)$ from now on will be dropped.  Parameters $w_{ij}(t)$ describe functional couplings between neurons and correspond to the functional connectivity in GLM.  The rate-function $f(J)$ is exponential, $f(J)=\exp(J)$.  This choice is motivated by significant computational simplifications when estimating GLM parameters, by making MAP estimation problem convex - an important advantage for working with large volumes of data
in neural activity recordings \cite{PAN03d}.

If actual neural spikes $n_i(t)$ were directly observed, the problem of estimating functional connectivity from a set of observed spike trains would have reduces to a known problem such as in \cite{PILL07}. With calcium imaging, however, we do not directly observe spike trains. Instead, fluorescent signal from the calcium-reporter couples with the neural activity via hidden nonlinear calcium dynamics \cite{Vogelstein2009}, 

\begin{equation}\label{eqn:ca:definition}
\begin{array}{l}
C_i(t) = C_i(t-1) + (C^i_b-C_i(t-1)) \Delta/\tau^i_c + A^i_c n_i(t)+\epsilon_c^i(t), \\
F_i(t) \sim \text{Poiss}( S(C_i(t);\alpha_i, \beta_i) ).
\end{array}
\end{equation}

Eq. \eqref{eqn:ca:definition} describes evolution of intracellular calcium concentration $C_i(t)$ in the neuron $i$ with time $t$. Under normal conditions, the intracellular concentration is fluctuating around the baseline level of $C^i_b$ with normally distributed noise $\epsilon^i_c(t)$ with variance $(\sigma^i_c)^2\Delta$, first two and the last terms.  The third term describes instantaneous jump in the intracellular calcium concentration $A^i_c$ when the neuron fires a spike.  Calcium imaging signal, $F_i(t)$, is described by the count of photons collected at the detector per neural cell per one imaging frame. It is distributed according to Poisson statistics with the mean given by generalized Hill function $S(C)=\alpha_i C/(C+K_d) + \beta_i$ \cite{Yasuda2004}.

The hierarchical model Eq. \eqref{eqn:glm:definition} and \eqref{eqn:ca:definition} is governed by $6N$ calcium dynamics parameters - calcium decay time-constant $\tau_c$, background calcium concentration $C_{b}$, fluctuations in calcium concentration $\sigma_c$, per-spike calcium jump $A_c$, photon budget scaling $\alpha_c$ and background $\beta_c$, which we jointly denote as $M_i=\{\tau^i_c, C^i_{b}, \sigma^i_c, A^i_c, \alpha_i, \beta_i\}$, and $m N^2 + N$ functional connectivity weights $w_{ij}(t)$ and $b^0_i$ which we jointly denote as $W=\{w_{ij}(t), b_i\}$, $m$ is the temporal depth of coupling weights $w_{ij}(t)$.


\begin{table}[h!b!p!]
\caption{Table of notations. Indices $i, j=1\ldots N$ refer to the neurons in the population. Bold notation is used to represent the vector states over all neurons, e.g. ${\bf X}=\{X_i, i=1\ldots N\}$. Whenever $t$ is dropped for brevity the entire sequence $t=1\ldots T$ is implied, e.g. ${\bf X}=\{{\bf X}(t), t=1\ldots T\}$.}
\label{table:notation}
\begin{tabular}{ll}
Number of neurons         & $N$ \\
Calcium imaging duration & $T$ \\
Single neuron spike & $n_i(t)$ \\
Single neuron calcium concentration & $C_i(t)$ \\
Single neuron state & $X_i(t)=(n_i(t), C_i(t))$ \\
Sequence of neural states & $X_i=\{X_i(t), t=1\ldots T\}$ \\
Fluorescence observation from single neuron & $F_i(t)$ \\
Sequence of fluorescence observations & $F_i=\{F_i(t), t=1\ldots T\}$ \\
Set of spike trains from all neurons   & ${\bf X}=\{X_i\}$ \\
Set of fluorescence observations from all neurons & ${\bf F}=\{F_i\}$ \\
Functional connection weights & $w_{ij}(t)$ \\
Matrix of all connection weights & $W=\{w_{ij}(t)\}$ \\
Calcium dynamics model for single neuron & $M_i$ \\
Calcium dynamics model for all neurons & $M=\{M_i\}$
\end{tabular}
\end{table}