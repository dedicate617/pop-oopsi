Deducing the structure of neural circuits from neural activity is one of the central problems of modern neuroscience. Recent studies have examined this problem in the context of neural activity observations for dozens of neurons using multi-electrode techniques. Here, we present a model-based optimal approach for inferring structure of neural micro-circuits containing hundreds of neurons observed optically with calcium imaging. While calcium imaging allows less intrusive observations of neural activity in much larger populations than multi-electrode techniques, the images are less direct observations of neural spike trains, and have limited time resolution and signal quality.  To infer the functional connectivity matrix of observable micro-circuits from such indirect data, we assume a coupled hidden-Markov chain model, where each neuron is a generalized linear model.  The sufficient statistics of the cross-coupling terms, which comprise the functional connectivity matrix, are obtained using an embedded-chain-within-blockwise-Gibbs algorithm for jointly sampling spike trains, given the calcium traces. By utilizing a factorized approximation, we implement our algorithm in parallel on a high-performance cluster, without a significant degradation of our results.  Furthermore, by imposing biophysically realistic constraints on our model, such as a sparse constraint on the functional connectivity matrix, we can reduce the amount of data required to obtain a good fit.  In realistic simulations, we show that our method can successfully infer connectivity patterns from $\sim 10$ minutes of calcium imaging data for neural populations up to $500$ neurons large, in only $\sim 10$ minutes of computation time.  Finally, we show the robustness of our method to various model misspecifications.  Thus, this approach seems ready to be utilized by experimental neuroscientists to unravel the functional connection matrices underlying behavior and perception.
