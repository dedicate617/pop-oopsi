Since Ramon y Cajal discovered that the brain is a rich and dense \emph{network} of neurons \cite{RamonyCajal04,RamonyCajal23}, neuroscientists have been intensely curious about the details of these networks, which are believed to be the biological substrate for memory, cognition, and perception. While we have learned a great deal in the last century about ``macro-circuits'' --- the connectivity between coarsely-defined brain areas --- relatively little is known about ``micro-circuits,'' i.e., the connectivity within populations of neurons at a fine-grained cellular level. One can imagine two complementary strategies for inferring micro-circuits: anatomical and functional. Anatomical approaches to inferring circuitry do not rely on observing neural activity.  Several interesting anatomical strategies are being pursued by other groups , such as array tomography \cite{MichevaSmith07}, ``brainbow'' mice \cite{Brainbow07}, and serial electron microscopy \cite{Briggman2006}. Our work,  takes a functional approach: our aim is to infer micro-circuits by observing the activity of a population of neurons.

Experimental tools that enable approximately simultaneous observations of the activity of many (e.g., $O(10^3)$) neurons are now widely available. While arrays of extracellular electrodes have been exploited for this purpose, the arrays most often used \emph{in vivo} are inadequate for inferring monosynaptic connectivity in very large populations of neurons, as the inter-electrode spacing is typically too large to record from neighboring neurons, and technologically possible number of electrodes in the arrays limits the number of neurons that can be simultaneously observed \cite{HATS98,HARR03,Stein04,Santhanam06,Harris07}\footnote{It is worth noting, however, that multielectrode arrays which have been recently developed for use in the retina are capable of much denser sampling \cite{Berry2004,Litke2004,Petrusca07,PILL07}.}. Alternately, calcium-sensitive fluorescent indicators allow us to observe the spiking activity of up to $O(10^4)$ neighboring neurons \cite{Tsien89}.  Importantly, neighboring neurons are more likely connected to one another than than distant neurons \cite{Abeles91,Braitenberg1998}. Some organic dyes achieve sufficiently high signal-to-noise ratios (SNR) that individual action potentials (spikes) may be resolved \cite{ImagingManual}, and bulk-loading techniques enable experimentalists to simultaneously fill populations of neurons with such dyes \cite{StosiekKonnerth03}. In addition, genetically encoded calcium indicators are under rapid development in a number of groups, and are approaching SNR levels of nearly single spike accuracy as well \cite{WallaceHasan08}. Microscopy technologies for collecting fluorescence signals are also rapidly developing. Cooled CCDs for wide-field imaging (either epifluorscence or confocal) now achieve a quantum efficiency of $\approx 90 \%$ with frame rates up to $60$ Hz or greater, depending on the width of the field of view \cite{Djurisic04}. For in vivo work, 2-photon laser scanning microscopy can achieve similar frame rates, using either acoustic-optical deflectors to focus light at arbitrary locations in three-dimensional space \cite{ReddySaggau05,Iyer06,SalomeBourdieu06,ReddySaggau08}, or resonant scanners \cite{NguyenParker01}. Together, these experimental tools can provide movies of calcium fluorescence transients for small populations of neurons (e.g., $O(10^2)$) with ``reasonable'' SNR, and at imaging frequencies of $30$ Hz or greater, in both the in vitro and in vivo settings.

Given these experimental advances in functional neural imaging, our goal is to develop efficient computational and statistical methods to exploit this data for the analysis of neural connectivity (see Fig.~\ref{fig:data_schematic} for a schematic overview). One major challenge here is that calcium transients due to action potentials provide indirect observation of the neural activity, and decay about an order of magnitude slower than the time course of the underlying neural activity \cite{ImagingManual}. Thus, to properly analyze the functional network connectivity, we must incorporate a method for effectively deconvolving the observed noisy fluorescence signal to obtain estimates of the underlying spiking rates \cite{YaksiFriedrich06,GreenbergKerr08,Vogelstein2009}. To this end we introduce a coupled Markovian state-space model that relates the observed variables (fluorescence traces from the neurons in the microscope's field of view) to the hidden variables (spike trains and intracellular calcium concentrations of these neurons), as governed by a set of biophysical parameters including the network connectivity matrix. Given this model, we derive a Monte Carlo Expectation Maximization algorithm for obtaining the maximum a posteriori estimates of the parameters of interest. Standard Monte Carlo sampling procedures (e.g., Gibbs sampling or sequential Monte Carlo) are inadequate in this setting, due to the high dimensionality and non-linear, non-Gaussian dynamics of the hidden variables in our model; we therefore develop a specialized blockwise-Gibbs approach to overcome these obstacles. This strategy enables us to accurately infer the functional connectivity matrix from large simulated neural populations, under realistic assumptions about the dynamics and observation parameters. We describe our approach below, along with several methods for improving its computational speed and statistical efficiency. 