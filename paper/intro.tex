Since Ramon y Cajal discovered that the brain is a rich and dense \emph{network} of neurons, neuroscientists have wondered about the details of these networks.  Since then, while much has been learned about ``macro-circuits''  --- the connectivity between populations of neurons --- relatively little is known about ``micro-circuits'' --- the connectivity within populations of neurons. Broadly, one can imagine two distinct strategies for inferring microcircuit connectivity: anatomical and functional.  Anatomical approaches to inferring circuitry include any strategy that does not consider the neural \emph{activity}.  Recently developed technologies including array tomography \cite{MichevaSmith07}, brainbow mice \cite{Brainbow07}, and serial electron microscopy based approaches \cite{Briggman2006}, are rapidly improving and show great promise.  This work takes a functional approach, i.e., our aim is to be able to infer the microcircuit by only observing the activity of a population of neurons.

Experimental tools that enable approximately simultaneous observations of the activity many (e.g., $O(10^3)$) neurons are now widely available.  While arrays of extracellular electrodes have been exploited for this purpose, those used in vivo are inadequate for inferring monosynaptic connectivity, as the inter-electrode spacing is typically too large (however, multi-electrode arrays in slice and retina do not share this problem, see for example \cite{PILL07}).  Alternately, calcium-sensitive fluorescent indicators provide a glimpse into the spiking activity of many neighboring neurons \cite{Tsien89}, which are more likely to be connected \cite{Abeles91, Braitenberg1998}. Some organic dyes achieve signal-to-noise ratios (SNRs) yielding single spike resolution \cite{ImagingManual}.  In combination with these dyes, bulk-loading techniques enable experimentalists to simultaneously fill populations of neurons with such dyes \cite{StosiekKonnerth03}.  In addition, genetically encoded calcium indicators are under rapid development from a number of groups, and are approaching SNR levels of nearly single spike accuracy as well \cite{WallaceHasan08}. Microscopy technologies for collecting fluorescence signals are also rapidly developing.  Cooled CCDs for wide-field imaging (either epifluorscence or confocal) now achieve a quantum efficiency of $\approx 90 \%$ with frame rates easily exceeding $30$ or $60$ Hz \cite{Djurisic04}.  For in vivo work, 2-photon laser scanning microscopy can achieve similar frame rates, by designing software to efficiently control the typical scanners (Valentin Naegerl, Tom Mrsic-Flogel, and Bruno Pichler, personal communications), using acoustic-optical deflectors to focus light at arbitrary locations in (three-dimensional) space \cite{ReddySaggau05, Iyer06, SalomeBourdieu06, ReddySaggau08}, or using resonant scanners \cite{NguyenParker01}.  Together, these experimental tools can provide movies indicating calcium based fluorescent transients for small populations of neurons (e.g., $O(10^2)$), with ``reasonable'' SNR, at 30 Hz, both in the in vitro and in vivo scenarios.  

Given these experimental advances in functional neural imaging, we aim here to develop complementary computational tools.  We first define a coupled hidden Markov Model, relating the observed variables (fluorescence traces from observable neurons) to the hidden variables (spike trains of those neurons), as governed by a small set of parameters, including the functional connectivity matrix.  Given this model, we derive an expectation maximization (EM) algorithm, to approximate the maximum a posteriori estimates of the parameters of interest.  Because our model is high dimensional, non-linear, and non-Gaussian, the E step must be approximated.  Standard sampling procedures, such as sequential monte carlo, are known to perform poorly in such scenarios \cite{DGL01}.  We therefore develop a novel embedded-chain-within-blockwise-Gibbs approach to overcome these obstacles \cite{Neal03}. This strategy enables us to accurately infer the functional connection matrix from a small (e.g., $\approx 200$) population of neurons, making realistic assumptions about dynamics and observation parameters.  By introducing a factorization approximation, we still perform nearly as well, and greatly reduce the computational burden.  Furthermore, imposing biophysically based priors leads to a comparable reduction in the amount of data required to obtain satisfactory estimates.  We then quantify the accuracy of our estimates as a function of imaging rate, noise, experimental duration, number of neurons, strong spiking correlations, and certain model misspecifications.  Given realistic assumptions about the dynamics and noise parameters, we show that 10 minutes of data imaged at 30 Hz from 200 neurons requires is sufficient for us to accurately estimate the functional connection matrix of the observable neurons, and only requires about 10 minutes of processing on a cluster.  