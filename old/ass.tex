\documentclass[12pt]{article}

\usepackage{amsmath}
\usepackage{graphicx}
\usepackage{amsfonts}
\usepackage{amsthm}
%\usepackage{cite}
\usepackage{times}
\usepackage{geometry}
\usepackage[round,numbers,sort&compress]{natbib}
\oddsidemargin=0.0in %%this makes the odd side margin go to the default of 1inch
\evensidemargin=0.0in
\textwidth=6.5in
\textheight=9in %%sets the textwidth to 6.5, which leaves 1 for the remaining right margin with 8 1/2X11inch paper

\providecommand{\ve}[1]{\boldsymbol{#1}}
\providecommand{\norm}[1]{\left \lVert#1 \right  \rVert}
\providecommand{\deter}[1]{\lvert #1 \rvert}
\providecommand{\abs}[1]{\lvert #1 \rvert}
\providecommand{\tran}{\mbox{${}^{\text{T}}$}}
\providecommand{\transpose}{\mbox{${}^{\text{T}}$}}
\providecommand{\ve}[1]{\boldsymbol{#1}}
\DeclareMathOperator*{\argmax}{argmax}
\DeclareMathOperator*{\argmin}{argmin}
\DeclareMathOperator*{\find}{find}

\newcommand{\thetn}{\ve{\theta}}
\newcommand{\theth}{\widehat{\ve{\theta}}}
\newcommand{\theto}{\ve{\theta}'}
\newcommand{\p}{P_{\thetn}}
\newcommand{\phat}{\widehat{P}_{\thetn}(F_v | \Ca_t)}
\newcommand{\pT}{P_{\thetn_{Tr}}} %\thetn_T
\newcommand{\pO}{P_{\thetn_o}} %\thetn_o
\newcommand{\Q}{Q(\thetn,\theto)}
\newcommand{\m}{m^{\ast}}
\newcommand{\q}{q\big(\ve{H}_t^{(i)}\big)}
\newcommand{\Ca}{[\text{Ca}^{2+}]}

%\usepackage[hypertex]{hyperref}    %for LaTeX
\usepackage{hyperref}               %for pdfLaTeX

\title{SMC-MC}
\author{Joshua Vogelstein, some fuckers,  Liam Paninski}

\begin{document}

\maketitle

%\begin{abstract}
%
%\end{abstract}
%

start by assuming only one neuron


\begin{enumerate}
\item generate N trajectories using a single forward pass of the particle filter (using prior sampler at first)
\item add the most recently selected path (ie, ``current path''), making us have N+1 trajectories
\item restrict model to discrete spaces. 
\item compute q(x), the prob of sampling the trajectory, 'x'.  we can compute this using rabiner's forward-backward approach
\item sample backward using $Z  P(h_{t+1} | h_t) P(h_t | y_{1:t})$ (ie, product of transition prob and particle weight).  this is O(N+1)
\item compute M-H prob of acceptance: $r=q(y) p(x) / [q(x) p(y)]$, where $x$ is newly sampled trajectory, $y$ is the current path, $q(\cdot)$ is the probability of sampling (computed using (4)), and $p(x)$ is secretly $p(x)p(D|x)$, ie, the prior prob times the likelihood of the data given the trajectory
\item sample $p \sim U(0,1)$, if $p< \min(1,r)$, accept new sequence (these steps comprise the E step)
\item upon convergence, estimate parameters (ie, M step)
\end{enumerate}

once we have that working, we extend the framework to facilitate multiple neurons.  in this case, the probability of spiking (which we use as a prior to sample in the particle filter), changes from  

\begin{equation}
\lambda(t) = f(b+ \ve{k}' \ve{x}_t + \sum_{s<t} \ve{\omega} \ve{h}_s)
\end{equation}

to

\begin{equation}
\lambda_i(t) = f(b + \ve{k}' \ve{x}_t + \sum_{s<t} \ve{\omega} \ve{h}_s + \sum_{u>t} \ve{\zeta} \ve{g}_u)
\end{equation}

where $\ve{h}$ corresponds to spike histories from all neurons, and $\ve{g}$ corresponds to spike futures from all neurons other than $i$, a la \cite{PillowLatham07}.

we then iterate all but the last step (ie, the M step) above for neuron $i$, holding the other spike trains fixed, ie, we generate samples from $\p(n_i(t) | n_{\backslash i}(t), \ve{O}_{1:t})$ in the particle filtering step.

as this probably will take a while, we can do a number of things to speed it up

\begin{itemize}
\item skip generating particles step sometimes
\item do an increment/decrement thing, where with each E step, we select a trajectory to drop, and replace it with the newly accepted trajectory
\item use a dual-tree approach (or other N-body style approximation) for some of the stuff up there
\item update parameters within each E step by using the approach described \cite{NgDearden05}
\item instead of backward sampling, use Viterbi for PF (as described in \cite{GodsillDoucet01}, or the fast version of that approach (as described in \cite{KlaasFrietas05}
\end{itemize}


\end{document}
